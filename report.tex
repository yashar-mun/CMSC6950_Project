\documentclass[12pt]{article}
\usepackage{graphicx}
\usepackage{subcaption}
\usepackage{hyperref}
\usepackage{float}
\usepackage{mathtools}

\title{CMSC 6950 Project}
\date{June 2021}
\author{Yashar Tavakoli} 

\setlength\parindent{0pt}
\begin{document}
\maketitle
\section{Introduction}

Argo is the name of 
an international data-collection program in oceanography. 
The program consists of a fleet of instruments around the globe, 
which operate
autonomously to measure and report different ocean variables. 
In this report, the name argo refer to both the program, and the
instruments.
Argo data is massive and can be nonhomogeneous. Thanks
to argopy (which is a light-weight
python library) however, accessing argo data can be quite hassle-free. The mentioned
library, also provides data manipulation and visualization
facilities. \\

This project consists of two computational tasks, which in a 
nutshell, aim at fetching data from \verb|argopy|, 
manipulating the data a bit, and presenting the results
through proper visualization. In my experience, out-of-the-box
visualization provisions of argopy, is quite primitive. So I have 
employed more powerful python libraries for visualization. \\

Before diving in, here is the workflow of this project 
(each python script is responsible for a single computational task):

\begin{equation*}
    \begin{rcases}
    \text{2d\underline{\hspace{8pt}}density\underline{\hspace{8pt}}argos\underline{\hspace{8pt}}locations.py}\hspace{14pt}\begin{cases}
      \text{hexbin\underline{\hspace{8pt}}argos\underline{\hspace{8pt}}locations.png}
    \end{cases}
    \\
    \text{correlation\underline{\hspace{8pt}}between\underline{\hspace{8pt}}variables.py}\begin{cases}
        \text{map\underline{\hspace{8pt}}of\underline{\hspace{8pt}}locations.png}\\
        \text{correlation1.png}\\
        \text{correlation2.png}\\
        \text{correlation3.png}\\
        \text{correlation4.png}
    \end{cases}
    \end{rcases}
    \text{report.pdf}
\end{equation*}

Now without further further adieu, I will explain the two computational tasks I
took on.

\section{Density of argos' locations}

Every argo during its lifetime, visits a sequence of 
locations (at different depths). Each location during a time-period
therefore, can be visited by any number of argos. Provided that the location
does not point to land of course. Now an interesting question one can ask, would
be: how frequent a location is visited by argos during a certain time-period?
Or more interestingly, one can look at a certain geographical region instead.
A scatter-plot over a longitude-latitude box, would be a quick 
approach to answer the mentioned question. That said, 
scatter-plots suffer from certain shortcomings:

\begin{itemize}
    \item When the data is dense, a scatter plot would be messy and less interpret-able to the eye. 
    \item There is no visual component accompanying scatter-plots, whereby one can learn the number of points in a given area.
\end{itemize}

A workaround in this situation, would be some sort of 2-dimensional 
\textit{destiny plot}, whereby
the distribution of data is more readily observable. \\

One of the more sophisticated density plots perhaps, is 
\verb|seaborn.kdeplot|. 
But seaborn is not integrated with \verb|mpl_toolkits.basemap|,
which is the library I will use for the underlay geographical 
map (a map-less plot would also suffer from poor interpret-ability). 
Instead of \verb|seaborn.kdeplot|, I will use 
\verb|matplotlib.pyplot.hexbin| which
is well integrated with \verb|basemap|, and provides comparable results.\\

To fetch geographically bounded argo data in a given time-period:
 \begin{itemize}
     \item One can use \verb|argo_loader.region().to_xarray()| which takes longitude, latitude, and time period as arguments, and 
     returns the argo data in multidimensional \verb|xarray|.
     \item Since working
     with Pandas (2-dimensional) data-frames is more straightforward, 
     I have flattened the \verb|xarrays| 
     (using \verb|argo.point2profile()|),
     and converted them to Panda's data-frames 
     (using \verb|to_dataframe()|).
     \item Flattening a multidimensional \verb|xarray| to 
     2-dimensional produces a multi-index data-frame. 
     In computational task no.1, 
     we do not need any of the indices so I have discarded 
     the indices (using \verb|reset_index()|). Furthermore I 
     have dismissed all
     the columns but longitude and latitude. 
     Lastly, before 
     drawing the \verb|matplotlib.pyplot.hexbin| on the map, I have
     prepared the data for \verb|mpl_toolkits.basemap|.
 \end{itemize}

 As depicted in Fig.\ref{hexbin}, the region I have chosen for 
 this task, stretches from 140 to 150 
 in longitude, and from 35 to 50 in latitude (around Japan). 
 The time-frame is chosen to be a six months period between 
 2015-06-01 and 2015-12-30. The image depicts the result. One
 last step I performed as manifest in the image, 
 is highlight some locations on the map 
 to be able to better explain the plot: It seems to be
 fair to say that argos cover offshore more frequently 
 than ports such as Sendai. One exception perhaps, would be
 off the coast of Sapporo (in Hokkaido) with more than 1000 
 reports. One other curious fact perhaps, is argos' low frequency around 
 north of Kunashiri.    

\begin{figure}[!h]
\centering
\includegraphics[scale=0.9]{hexbin_argos_locations.png}
\caption{Hexbin plot of argo data}
\label{hexbin}
\end{figure}

\section{A correlation analysis of Level, Pressure, Salinity, 
and Temperature.}

A question one might ask regarding argo data, would be the correlation 
between each pair in \{Level, 
Pressure, Salinity, Temperature\}. To keep the computations light, 
I have singled out four profiles from four different
argos located in Indian Ocean, South Atlantic, North Atlantic, and 
Pacific (using the service provided by 
\href{https://fleetmonitoring.euro-argo.eu/dashboard?Status=Active}{fleetmonitoring.euro-argo}). 
To fetch  
the data, I have used \verb|argo_loader.profile().to_xarray()|, which 
takes in the argo WMO and profile numbers, and returns the data
in the form of four different \verb|xarrays| (I have chosen profile 
numbers 
manually, so that the produced plots show decent diversity).\\

Next, as explained in task no.1, I have flattened the \verb|xarrays| 
and converted them to Panda's data-frames. Here, unlike
task no.1, the produced indices in the data-frames are useful: 
the Level variable in the data-frames, is defined as 
index. As the first step, I have reintroduced Level as a column and 
dismissed all the column except Level, Pressure, 
Salinity, and Temperature.\\

Before moving forward, let's take a look at a map produced by 
\verb|basemap| hinting at the spots where
the chosen argos resided or have resided (Fig.\ref{map}). 
Since each profile consists of a sequence of longitudes and 
latitudes,
I have taken a mean to get an idea about the whereabouts of 
the argos pertaining to the selected profiles.\\

\begin{figure}[!ht]
    \centering
    \includegraphics[scale=0.7]{map_of_locations.png}
    \caption{Locations of the argos}
    \label{map}
\end{figure}

Now back to the main task at hand, we need a proper 
visualization which help us get an idea of the correlation
between the variables in question. To do this, I have 
devised a composite plot from three components: 

\begin{figure}[!ht]
    \begin{tabular}{cc}
        \hspace{-30pt} \includegraphics[width=70mm]{correlation1.png} &\hspace{10pt}   \includegraphics[width=70mm]{correlation2.png} \\
        (a) Location: Indian Ocean & (b) Location: North Pacific \\[15pt]
        \hspace{-30pt} \includegraphics[width=70mm]{correlation3.png} &\hspace{10pt}   \includegraphics[width=70mm]{correlation4.png} \\
        (c) Location: South Atlantic & (d) Location: North Atlantic \\[15pt]
        \end{tabular}
        \caption{Correlation between the variables}
        \label{corr}
\end{figure}

\begin{enumerate}
    \item The lower-triangle consists of pairwise scatter-plots.
    \item The diagonal consists of histogram bars illustrating 
    the distribution of variables.
    \item The upper-triangle indicates the Pearson's r 
    coefficient for each pair of variables 
    (implemented though \verb|reg_coef()| 
    in the code). The Pearson's r ranges from 1 to -1.
    A value close to 1 (-1) indicates a (reverse) near
    perfect linear relationship between the variables.
    A value close to 0, implies that there is almost no 
    linear correlation between the variables.
\end{enumerate}



Now lets take a couple of notes regarding the
correlation between the variables as illustrated in
Fig.\ref{corr}:
    
\begin{itemize}
    \item The scatter plots and coefficients for all the cases 
    indicate a near perfect
    linear relationship between Pressure and Level. 
    \item All the plots also demonstrate that there exist a 
    significant reverse linear relationship between Temperature 
    and Level,
    as well as Temperature and Pressure.
    \item The rest of the pair of variables show non-conclusive 
    behavior. The situation here perhaps, would entail a more educated 
    investigation. That said, curiously there is no near-zero 
    coefficient for any of the pairs.
    \item The variables are similarly correlated pertaining to 
    profiles from Indian Ocean, and South Atlantic. Same with North Atlantic
    and North Pacific. Hence a crude non-educated guess: the divide 
    is between northern and southern hemisphere (?).
\end{itemize}

\section{Conclusion}
In this project, I implemented two computational tasks involving argopy, where
data is fetched from external sources and visualized properly through python plotting
libraries.

\begin{thebibliography}{9}
    \bibitem{latexcompanion} 
    Maze et al.,
    \textit{argopy: A Python library for Argo ocean data analysis.}. 
    Journal of Open Source Software, 5(53), 2425, 2020.
\end{thebibliography}

\end{document}